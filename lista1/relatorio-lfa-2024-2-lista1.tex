\documentclass[a4paper, 11pt]{scrartcl}

\usepackage[utf8]{inputenc}

\title{Lista 1 --- Respostas}
\subtitle{Linguagens Formais de Autômatos}
\author{Henrique de Souza Lima}
\date{2024/2}

\begin{document}

\maketitle

\section*{Problema 1}

\subsection*{Enunciado}
Escreva um programa que converte um número inteiro de uma base numérica para outra. O
programa deve receber como entrada o número a ser convertido, a base original e a base para a
qual ele deve ser convertido. As bases suportadas devem ser 2 (binária), 8 (octal), 10 (decimal),
12 (duodecimal), 16 (hexadecimal), e 20 (vigesimal).1 (rodapé: 1Para base numéricas maiores do que 10, 
use as letras maiúsculas no alfabeto latino como os dígitos adicionais. Por exemplo, 
os dígito da base 16 serão {0, 1, 2, 3, 4, 5, 6, 7, 8, 9, A, B, C, D, E, F}.)

\subsection*{Solução}
A solução adotada foi implementar 2 funções, sendo uma auxiliar com uso de recursão. 
Primeiramente, converte-se a string recebida por um número inteiro, multiplicando
cada dígito pelo seu valor correspondente, associado através de um Map(Char, Int),
e da posição que ocupa, invertendo a string recebida para que os indices estejam corretamente
relacionados com o expoente daquela base; por fim, soma-se todos os valores obtidos. Com o
resultado, chama-se a função recursiva de cauda com o numero e sua base final, cujo caso base é
quando o número em seu parametro é igual a 0. Nos demais casos, o número é dividido 
sucessivamente pela base desejada e possui o caracter associado ao o resto desta divisão acrescentado 
ao início de uma string, iniciada vazia. Ao fim, obtém-se a convertido para a base desejada.  

\section*{Problema 2}

\subsection*{Enunciado}
Desenvolva um programa que verifica se uma expressão composta apenas de parênteses (‘(’ e
‘)’) está corretamente balanceada, ou seja, se cada parêntese de abertura tem um correspondente
parêntese de fechamento. O programa receberá como entrada uma string contendo apenas
parênteses, e como saída uma mensagem que indica se a expressão está bem formada.

\subsection*{Solução}
A solução adotada foi implementar uma recursão. O caso base é quando a string é
completamente percorrida ou quando o número de parêntesis abertos torna-se negativo.
Caso tenha sido completamente percorrida, verifica se o contador de parentêsis 
abertos (rest, passado como parametro) é igual a 0, indicando que cada parentêsis 
aberto (+1 ao contador) possui correspondente  fechado (-1 ao contador), o qual mostra
tratar-se de uma expressão bem formada. Caso contrário, envia mensagens adequadas
à razão da falta de correspondência. O contador que percorre os caracteres da string
inicia-se com 0 e é passado como parametro, sendo incrementado a cada iteração.

\section*{Problema 3}

\subsection*{Enunciado}
Escreva um programa que verifique se um número é perfeito. Um número perfeito é aquele
que é igual à soma de seus divisores próprios, i.e., os divisores excluindo o um o próprio número.
O programa receberá como entrada um número inteiro positivo, e como saída uma mensagem
que indica se o número é perfeito ou não.

\subsection*{Solução}
A solução adotada foi implementar uma recursão. O caso base é quando o divisor atual, 
inicialmente igual a 1, é igual ou maior que o número passado inicialmente para a função. 
Caso seja, verifica se a soma dos divisores, positivos, do número desejado, exceto ele mesmo, é
igual ao próprio número, e indica, em uma mensagem, se é ou não perfeito em conformidade
com esta condição. O conjunto dos divisores, implementado como set, inicia-se 
vazio. Nos demais casos, verifica-se se o resto da razão inteira entre do número e o
divisor atual é igual ou não a zero, e acrescentará este divisor ao set em caso afirmativo;
em seguida, soma o valor do divisor atual por 1. O resultado é um set contendo todos
os divisores desse número.

\section*{Problema 4}

\subsection*{Enunciado}
Crie um programa que calcula o valor de uma expressão da álgebra de conjuntos. O programa
deve permitir ao usuário inserir conjuntos (apenas conjuntos definidos por extensão, i.e., listando
todos os elementos) e, em seguida, o programa deve receber como entrada uma string que
representa a expressão e retornar o resultado. A expressão em questão poderá usar:
\begin{itemize}
    \item União ($A \cup B$);
    \item Interseção ($A \cap B$);
    \item Diferença ($A - B$);
    \item Diferença simétrica ($A \Delta B$);
    \item Complemento ($\sim A$);
    \item Produto cartesiano ($A \times B$);
    \item Conjunto das partes ($P(A)$).
\end{itemize}
Exemplos simples:
\begin{itemize}
    \item $A \cup (B \cap C)$: União do conjunto $A$ com a interseção dos conjuntos $B$ e $C$.
    \item $(A - B) \Delta C$: Diferença simétrica entre a diferença de $A$ e $B$ e o conjunto $C$.
    \item $\sim(A \cup B)$: Complemento da união dos conjuntos $A$ e $B$.
    \item $P(A) - P(B)$: Diferença entre o conjunto das partes de $A$ e o conjunto das partes de $B$.
    \item $A \times (B \cup C)$: Produto cartesiano de $A$ com a união de $B$ e $C$.
\end{itemize}
Exemplos mais complexos:
\begin{itemize}
    \item $\sim((A \cap B) \cup (C - D))$: Complemento da união entre a interseção de $A$ e $B$ e a diferença
de $C$ e $D$.
    \item $P(A \cap B) \Delta P(A \cup B)$: Diferença simétrica entre o conjunto das partes da interseção de
$A$ e $B$ e o conjunto das partes da união de $A$ e $B$.
    \item $(A \times B) \cap (A \times C)$: Interseção do produto cartesiano de $A$ e $B$ com o produto cartesiano
de $A$ e $C$.
    \item $\sim(\sim A \cap \sim B)$: Complemento da interseção dos complementos de $A$ e $B$ (equivalente a $A \cup B$
pela Lei de De Morgan).
\end{itemize}
(O conjunto universo foi definido como a união de todos os conjuntos definidos.)

\subsection*{Solução}
A solução adotada foi adotar uma classe SetAlgebra que contém um traço (sealed trait) chamado SetResult, que é
estendido por StringSet e PowerSet, para permitir resultados em forma de Sets de strings ou Sets de Sets de Strings, como P(A). 
O programa possui funções para cada operação de conjuntos, sendo chamadas de acordo com a ocorrência de caracteres definidos, 
(Diferença, Conjunto das Partes, etc.), cuja assinatura, em inglês, corresponde ao seu produto. A função 
\texttt{evaluate\_expression} analisa uma expressão em notação de infixos a partir recebida como uma String, que é convertida em uma
lista de tokens e avaliada segundo a precedência devida. Para cada token, se for um conjunto definido pelo usuário, 
ele é empilhado; se for um operador, as operações correspondentes são aplicadas aos conjuntos no topo da pilha. 
Para expressões mais complexas, fez-se utilização de uma pilha de operadores de modo que as operações ocorram na ordem devida
para fazer a aplicação das operações das expressões entre parênteses. 
Os parâmetros são pedidos no o programa principal. Primeiro, a quantidade de conjuntos (até 4, nomeados de A a D); em seguida, 
lerá uma linha dos elementos do conjunto respectivo -- obtidos através da separação por espaço em branco entre 
caracteres (um split), seguido de toSet.


\end{document}
